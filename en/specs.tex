\section{Specifications}
\label{sec:Specifications}

\newcounter{rulei}[subsection]
\newcommand{\rcnii}{\stepcounter{rulei}\arabic{section}.\arabic{subsection}.\arabic{rulei}}
\renewcommand{\labelenumi}{\rcnii}

\subsection{Markers}
\label{sub:markers}
The arena, tokens, pedestals, and robots involved in the game are labelled with \textit{libkoki} markers.
Each marker pattern encodes a number.
Each marker number is associated with a particular feature within the arena, and also has an associated size.
The marker numbers and sizes are as follows:

\begin{center}
  \begin{tabular}{lcc}
    \toprule
    \textbf{Item} & \textbf{Marker Numbers} & \textbf{Marker Size (mm)} \\
    \midrule
    Arena boundary & 0 -- 27 & 250 \\
    Robots & 28 -- 31 & 100 \\
    Pedestals & 32 -- 40 & 200 \\
    Tokens & 41 -- 64 & 200 \\
    \bottomrule
  \end{tabular}
\end{center}

Two sets of marker codes will be used: one for development purpose, and one for the competition itself.
The competition set is only to be used inside the Student Robotics arena at the Student Robotics competition.
This is so that people carrying markers past the arena do not confuse robots.
The competition codes are 100 above the development codes.
When run in competition mode (specifiable through the robot's GUI), the software provided by Student Robotics will subtract 100 from the detected marker codes, as well as ignore the development codes.

The markers can be printed on a black-and-white printer.
Marker designs can be downloaded from the documentation section of the Student Robotics website.

Unless specified otherwise, all markers described in this document are oriented vertically such that the principle corner of the marker (which is indicated by a dark grey dot in the black marker border) is on the higher edge.

\subsection{Robot Badges}
\label{sec:robot-badges}

\begin{figure}
  \centering
  \includegraphics[width=\textwidth]{./images/robot-marker.pdf}
  \caption{An example robot badge.
           The blue areas shown are the human-compatible areas.
           All dimensions are in millimetres.}
  \label{fig:example-badge}
\end{figure}

\begin{enumerate}
\item A ``robot badge'' is a removable identifier that will be attached to a robot throughout a match.
      It features the robot's assigned marker for the match, as well as human-compatible areas to allow spectators to easily associate a robot with its starting location.
      An example of one of these badges is shown in figure~\ref{fig:example-badge}.
      The markings in the human-compatible areas are intentionally not specified.

\item A robot must feature four of the badge mounts shown in figure~\ref{fig:badge-mounting}.
      These mounts must permit a flat $200 \times 100mm$ panel to be attached to them.
      The three areas of each mount must feature the illustrated areas of hook-type Velcro to allow this panel to be fitted.

\item The four badge mounts must be on the exterior of the robot, parallel with the vertical plane, and should be perpendicular to each other about the vertical axis\footnote{Teams can apply for a team-specific rule alteration to the required number of badges.
      Clear justification must be provided by the team with such a request.}
      The orientation of the badge mounts is unimportant, but teams are encouraged to position them horizontally as shown in figure~\ref{fig:example-badge}.

\item The mapping between a given robot and its robot badge is as follows:

\begin{center}
  \begin{tabular}{cc}
    \toprule
    \textbf{Zone} & \textbf{Marker Number} \\
    \midrule
    0 (NW) & 28 \\
    1 (NE) & 29 \\
    2 (SE) & 30 \\
    3 (SW) & 31 \\
    \bottomrule
  \end{tabular}
\end{center}

\begin{figure}
  \centering
  \includegraphics[width=\textwidth]{./images/badge-mounting.pdf}
  \caption{The dimensions of the required robot badge mountings.
           The shaded areas are hook-type Velcro.
           All dimensions are in millimetres.}
  \label{fig:badge-mounting}
\end{figure}

\end{enumerate}

\subsection{Arena}
\label{sub:arena}
\begin{enumerate}
\item The match arena floor, overall, is an $8m \times 8m$ square, as shown in figure~\ref{fig:arena-dim}.
      The tolerance of these two dimensions is $\pm0.25m$.

\begin{figure}
  \centering
  \includegraphics[width=\textwidth]{./images/arena.pdf}
  \caption{\label{fig:arena-dim}A bird's-eye view of the arena.}
\end{figure}

\item The floor of the arena is carpeted with blue carpet tiles.
      The carpet tiles used in the arena are from B\&Q, with EAN 5014957151543.

\item The arena walls are $600\pm30mm$ high, the interior surfaces of which are white plastic-coated hardboard.

\item The arena features nine \textit{squares}.
      These areas are delineated by lines marked on the arena floor with $25mm$ wide paper-based masking tape.
      The naming of these squares is shown in figure~\ref{fig:arena-dim}.

\begin{figure}
  \centering
  \includegraphics[width=\textwidth]{./images/sidewall.pdf}
  \caption{Seven $250mm$ wide markers are spaced evenly along each $8m$ arena wall.
           The markers are placed $50mm$ above the floor.
	   All dimensions are in millimetres.}
  \label{fig:arena-wall}
\end{figure}

\begin{figure}
  \centering
  \includegraphics[width=0.5\textwidth]{./images/arena-markers.pdf}
  \caption{Twenty eight arena wall markers are positioned around the perimeter of the arena with the marker codes incrementing in an anti-clockwise fashion from the northern end of the west wall.
           The zones are counted from the north west corner in a clockwise fashion.}
  \label{fig:arena-zones}
\end{figure}

\item Each wall of the arena features seven $250mm$ libkoki markers.
      Figure~\ref{fig:arena-wall} shows the positioning of these markers, whilst figure~\ref{fig:arena-zones} shows the numbering of these markers.

\item Each robot will be assigned a zone number at the start of every match to indicate its starting position.
      The mapping of these zone numbers in the arena is shown in figure~\ref{fig:arena-zones}.

\item Student Robotics reserves the right to have up to three match officials in the arena during games.

\end{enumerate}

\subsection{Pedestals}
\label{sub:pedestals}
\begin{enumerate}
\item Pedestals are cuboid structures with base $320 \times 320mm \pm 15mm$.
      The top of a pedestal has a $20mm \pm 5mm$ high rim to prevent tokens falling off.
      The height of the pedestal, including the rim is $340mm \pm 15mm$.

\item Pedestals will be securely attached to the floor of the arena to prevent them being moved.

\item Each pedestal features a $200mm$ marker in the centre of each vertical side.

\item The mapping between pedestal position and marker is as follows:

\begin{center}
  \begin{tabular}{cc}
    \toprule
    \textbf{Position} & \textbf{Marker Number} \\
    \midrule
    NW & 32 \\
    N  & 33 \\
    NE & 34 \\
    W  & 35 \\
    C  & 36 \\
    E  & 37 \\
    SW & 38 \\
    S  & 39 \\
    SE & 40 \\
    \bottomrule
  \end{tabular}
\end{center}

\end{enumerate}

\subsection{Tokens}
\label{sub:Tokens}
\begin{enumerate}
\item Tokens are cubic corrugated cardboard boxes with side $305 \pm 15 mm$.
      \emph{Each team's kit contains two of these.}

\item Each token is associated with its own libkoki marker number and is labelled with six identical $200mm$ markers -- one on each face.

\item Tokens will be styled to match the human-compatible area of the robot badges on their associated robot, allowing spectators to follow game play.
      See section~\ref{sec:robot-badges}.

\item The mapping between a given robot and its tokens is as follows:

\begin{center}
  \begin{tabular}{cc}
    \toprule
    \textbf{Zone} & \textbf{Marker Numbers} \\
    \midrule
    0 (NW) & 41 -- 46 \\
    1 (NE) & 47 -- 52 \\
    2 (SE) & 53 -- 58 \\
    3 (SW) & 59 -- 64 \\
    \bottomrule
  \end{tabular}
\end{center}

\item The tokens belonging to a given robot will initially be positioned to the left of the robot as shown in figure~\ref{fig:token-position}.
      Teams have the option to place any one of their six tokens in/on their robot before the start of the match.

\begin{figure}
  \centering
  \includegraphics[width=\textwidth]{./images/token-position.pdf}
  \caption{Six $305 \pm 15mm$ wide tokens are spaced evenly $300 \pm 20mm$ to the left of the robot, along the arena wall.
           The tokens are placed $100 \pm 20mm$ away from each other and the edge of the arena.
           All dimensions are in millimetres.}
  \label{fig:token-position}
\end{figure}

\end{enumerate}

\clearpage
